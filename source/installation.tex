% Options for packages loaded elsewhere
\PassOptionsToPackage{unicode}{hyperref}
\PassOptionsToPackage{hyphens}{url}
\PassOptionsToPackage{dvipsnames,svgnames,x11names}{xcolor}
%
\documentclass[
  letterpaper,
  DIV=11,
  numbers=noendperiod]{scrreprt}

\usepackage{amsmath,amssymb}
\usepackage{iftex}
\ifPDFTeX
  \usepackage[T1]{fontenc}
  \usepackage[utf8]{inputenc}
  \usepackage{textcomp} % provide euro and other symbols
\else % if luatex or xetex
  \usepackage{unicode-math}
  \defaultfontfeatures{Scale=MatchLowercase}
  \defaultfontfeatures[\rmfamily]{Ligatures=TeX,Scale=1}
\fi
\usepackage{lmodern}
\ifPDFTeX\else  
    % xetex/luatex font selection
\fi
% Use upquote if available, for straight quotes in verbatim environments
\IfFileExists{upquote.sty}{\usepackage{upquote}}{}
\IfFileExists{microtype.sty}{% use microtype if available
  \usepackage[]{microtype}
  \UseMicrotypeSet[protrusion]{basicmath} % disable protrusion for tt fonts
}{}
\makeatletter
\@ifundefined{KOMAClassName}{% if non-KOMA class
  \IfFileExists{parskip.sty}{%
    \usepackage{parskip}
  }{% else
    \setlength{\parindent}{0pt}
    \setlength{\parskip}{6pt plus 2pt minus 1pt}}
}{% if KOMA class
  \KOMAoptions{parskip=half}}
\makeatother
\usepackage{xcolor}
\usepackage[heightrounded]{geometry}
\setlength{\emergencystretch}{3em} % prevent overfull lines
\setcounter{secnumdepth}{-\maxdimen} % remove section numbering
% Make \paragraph and \subparagraph free-standing
\ifx\paragraph\undefined\else
  \let\oldparagraph\paragraph
  \renewcommand{\paragraph}[1]{\oldparagraph{#1}\mbox{}}
\fi
\ifx\subparagraph\undefined\else
  \let\oldsubparagraph\subparagraph
  \renewcommand{\subparagraph}[1]{\oldsubparagraph{#1}\mbox{}}
\fi


\providecommand{\tightlist}{%
  \setlength{\itemsep}{0pt}\setlength{\parskip}{0pt}}\usepackage{longtable,booktabs,array}
\usepackage{calc} % for calculating minipage widths
% Correct order of tables after \paragraph or \subparagraph
\usepackage{etoolbox}
\makeatletter
\patchcmd\longtable{\par}{\if@noskipsec\mbox{}\fi\par}{}{}
\makeatother
% Allow footnotes in longtable head/foot
\IfFileExists{footnotehyper.sty}{\usepackage{footnotehyper}}{\usepackage{footnote}}
\makesavenoteenv{longtable}
\usepackage{graphicx}
\makeatletter
\def\maxwidth{\ifdim\Gin@nat@width>\linewidth\linewidth\else\Gin@nat@width\fi}
\def\maxheight{\ifdim\Gin@nat@height>\textheight\textheight\else\Gin@nat@height\fi}
\makeatother
% Scale images if necessary, so that they will not overflow the page
% margins by default, and it is still possible to overwrite the defaults
% using explicit options in \includegraphics[width, height, ...]{}
\setkeys{Gin}{width=\maxwidth,height=\maxheight,keepaspectratio}
% Set default figure placement to htbp
\makeatletter
\def\fps@figure{htbp}
\makeatother

\usepackage{fvextra}
\DefineVerbatimEnvironment{Highlighting}{Verbatim}{breaklines,commandchars=\\\{\}}
\KOMAoption{captions}{tableheading}
\makeatletter
\@ifpackageloaded{tcolorbox}{}{\usepackage[skins,breakable]{tcolorbox}}
\@ifpackageloaded{fontawesome5}{}{\usepackage{fontawesome5}}
\definecolor{quarto-callout-color}{HTML}{909090}
\definecolor{quarto-callout-note-color}{HTML}{0758E5}
\definecolor{quarto-callout-important-color}{HTML}{CC1914}
\definecolor{quarto-callout-warning-color}{HTML}{EB9113}
\definecolor{quarto-callout-tip-color}{HTML}{00A047}
\definecolor{quarto-callout-caution-color}{HTML}{FC5300}
\definecolor{quarto-callout-color-frame}{HTML}{acacac}
\definecolor{quarto-callout-note-color-frame}{HTML}{4582ec}
\definecolor{quarto-callout-important-color-frame}{HTML}{d9534f}
\definecolor{quarto-callout-warning-color-frame}{HTML}{f0ad4e}
\definecolor{quarto-callout-tip-color-frame}{HTML}{02b875}
\definecolor{quarto-callout-caution-color-frame}{HTML}{fd7e14}
\makeatother
\makeatletter
\@ifpackageloaded{caption}{}{\usepackage{caption}}
\AtBeginDocument{%
\ifdefined\contentsname
  \renewcommand*\contentsname{Table of contents}
\else
  \newcommand\contentsname{Table of contents}
\fi
\ifdefined\listfigurename
  \renewcommand*\listfigurename{List of Figures}
\else
  \newcommand\listfigurename{List of Figures}
\fi
\ifdefined\listtablename
  \renewcommand*\listtablename{List of Tables}
\else
  \newcommand\listtablename{List of Tables}
\fi
\ifdefined\figurename
  \renewcommand*\figurename{Figure}
\else
  \newcommand\figurename{Figure}
\fi
\ifdefined\tablename
  \renewcommand*\tablename{Table}
\else
  \newcommand\tablename{Table}
\fi
}
\@ifpackageloaded{float}{}{\usepackage{float}}
\floatstyle{ruled}
\@ifundefined{c@chapter}{\newfloat{codelisting}{h}{lop}}{\newfloat{codelisting}{h}{lop}[chapter]}
\floatname{codelisting}{Listing}
\newcommand*\listoflistings{\listof{codelisting}{List of Listings}}
\makeatother
\makeatletter
\makeatother
\makeatletter
\@ifpackageloaded{caption}{}{\usepackage{caption}}
\@ifpackageloaded{subcaption}{}{\usepackage{subcaption}}
\makeatother
\makeatletter
\@ifpackageloaded{tcolorbox}{}{\usepackage[skins,breakable]{tcolorbox}}
\makeatother
\makeatletter
\@ifundefined{shadecolor}{\definecolor{shadecolor}{HTML}{31BAE9}}{}
\makeatother
\makeatletter
\makeatother
\makeatletter
\ifdefined\Shaded\renewenvironment{Shaded}{\begin{tcolorbox}[borderline west={3pt}{0pt}{shadecolor}, interior hidden, enhanced, sharp corners, breakable, boxrule=0pt, frame hidden]}{\end{tcolorbox}}\fi
\makeatother
\ifLuaTeX
  \usepackage{selnolig}  % disable illegal ligatures
\fi
\usepackage{bookmark}

\IfFileExists{xurl.sty}{\usepackage{xurl}}{} % add URL line breaks if available
\urlstyle{same} % disable monospaced font for URLs
\hypersetup{
  colorlinks=true,
  linkcolor={blue},
  filecolor={Maroon},
  citecolor={Blue},
  urlcolor={Blue},
  pdfcreator={LaTeX via pandoc}}

\author{}
\date{}

\begin{document}

\RecustomVerbatimEnvironment{verbatim}{Verbatim}{
  showspaces = false,
  showtabs = false,
  breaksymbolleft={},
  breaklines
}

\renewcommand*\contentsname{Table of contents}
{
\hypersetup{linkcolor=}
\setcounter{tocdepth}{2}
\tableofcontents
}
\chapter{Setting up a terminal}\label{setting-up-a-terminal}

\section{Terminology}\label{terminology}

The \textbf{Linux command-line interface (CLI)} is an alternative to a
graphical user interface (GUI) with which you are likely more familiar.
Both interfaces allow you to interact with an operating system. The key
difference between the CLI and GUI is that the interaction with CLI is
based on issuing commands. In contrast, the interaction with a GUI
involves visual elements, such as windows, buttons, etc. CLI is often
also referred to as the shell, terminal, console, prompt or various
other names.

\textbf{Bash} is a type of interpreter that processes shell commands. A
shell interpreter takes commands in plain text format and calls the
operating system to do something, for example changing a directory or
modifying the content of some files. Bash itself stands for Bourne Again
Shell and it is one of the popular command-line shells used to run other
programs, many of which are useful for bioinformatic workflows. In this
tutorial, we assume that you work with a Bash shell.

\section{Installation guides}\label{installation-guides}

\subsection{Linux}\label{linux}

The default shell is usually Bash and there is no need to install
anything to be able to follow this tutorial. On most versions of Linux,
the shell accessible by running the Gnome Terminal or KDE Konsole or
xterm, which can be found via the applications menu or the search bar.
If your machine is set up to use something other than Bash, you should
be able to switch the shell by opening a terminal and typing
\texttt{bash}.

\subsection{Mac}\label{mac}

For Mac running macOS Mojave or earlier releases, the default Unix Shell
is Bash. For a Mac computer running macOS Catalina or later releases,
the default Unix Shell is Zsh. To open a terminal, try one or both of
the following:

\begin{itemize}
\tightlist
\item
  In Finder, select the Go menu, then select Utilities. Locate Terminal
  in the Utilities folder and open it.
\item
  Use the Mac `Spotlight' computer search function. Search for: Terminal
  and press Return.
\end{itemize}

To ensure that you work with a consistent shell and to check if your
machine is set up to use something other than Bash, type
\texttt{echo\ \$SHELL} in your terminal window. The name of the current
shell should be printed to the terminal window.

If your machine is set up to use something other than Bash, you can try
switching to Bash by opening a terminal and typing \texttt{bash}. To
check if that worked type \texttt{echo\ \$SHELL} again.

\subsection{Windows}\label{windows}

Operating systems like macOS and Linux come with a native command-line
terminal, making it straightforward to run bash commands. However, for
Windows users you need to install some software first to be able to use
bash, below you find three options:

One option to access the bash shell commands is using \textbf{Git Bash},
for detailed installation instructions please have a look at the
\href{https://carpentries.github.io/workshop-template/install_instructions/\#shell-windows}{carpenties
website}.

A second option is \textbf{Mobaxterm}, which enables Windows users to
execute basic Linux/Unix commands on their local machine, connect to an
HPC with SSH and to transfer files with SCP/SFTP (more on that later).
Installation instructions can be found
\href{https://hpc.ncsu.edu/Documents/mobaxterm.php}{here}.

A final option is to use Windows and Linux at the same time on a Windows
machine. The \textbf{Windows Subsystem for Linux (WSL2)} lets users
install a Linux distribution (such as Ubuntu, which is the default Linux
distribution, which we recommend to use) and use Linux applications,
utilities, and Bash command-line tools directly on Windows. This option
allows you to use all the tools available but since you more or less are
installing a separating system on your PC needs to have enough memory to
run this. Installation instructions can be found
\href{https://learn.microsoft.com/en-us/windows/wsl/install}{here}.

\begin{tcolorbox}[enhanced jigsaw, colback=white, breakable, coltitle=black, titlerule=0mm, toprule=.15mm, colframe=quarto-callout-note-color-frame, colbacktitle=quarto-callout-note-color!10!white, opacityback=0, bottomtitle=1mm, toptitle=1mm, rightrule=.15mm, opacitybacktitle=0.6, title=\textcolor{quarto-callout-note-color}{\faInfo}\hspace{0.5em}{Note}, arc=.35mm, leftrule=.75mm, bottomrule=.15mm, left=2mm]

I am myself mostly familiar with WSL and the following tutorial is
tailored towards the location of things when using WSL and Linux and
your folder structure might be slightly different when using Git Bash or
Mobaxterm.

Similarly, I am mainly familiar with the bash not the zsh shell. For Mac
users that have trouble switching to bash this might create some issues
when using wildcards but these users should be able to otherwise follow
most parts of this tutorial.

If parts of the tutorial do not work for you due to issues when working
with different operating systems/shells, feel free to contact me and I
can adjust the tutorial accordingly.

\end{tcolorbox}

\section{Sanity check}\label{sanity-check}

After you set everything up and opened a terminal you should see
something like this and are good to go if you want to follow the
tutorial:



\end{document}
