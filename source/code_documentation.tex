% Options for packages loaded elsewhere
\PassOptionsToPackage{unicode}{hyperref}
\PassOptionsToPackage{hyphens}{url}
\PassOptionsToPackage{dvipsnames,svgnames,x11names}{xcolor}
%
\documentclass[
  letterpaper,
  DIV=11,
  numbers=noendperiod]{scrreprt}

\usepackage{amsmath,amssymb}
\usepackage{iftex}
\ifPDFTeX
  \usepackage[T1]{fontenc}
  \usepackage[utf8]{inputenc}
  \usepackage{textcomp} % provide euro and other symbols
\else % if luatex or xetex
  \usepackage{unicode-math}
  \defaultfontfeatures{Scale=MatchLowercase}
  \defaultfontfeatures[\rmfamily]{Ligatures=TeX,Scale=1}
\fi
\usepackage{lmodern}
\ifPDFTeX\else  
    % xetex/luatex font selection
\fi
% Use upquote if available, for straight quotes in verbatim environments
\IfFileExists{upquote.sty}{\usepackage{upquote}}{}
\IfFileExists{microtype.sty}{% use microtype if available
  \usepackage[]{microtype}
  \UseMicrotypeSet[protrusion]{basicmath} % disable protrusion for tt fonts
}{}
\makeatletter
\@ifundefined{KOMAClassName}{% if non-KOMA class
  \IfFileExists{parskip.sty}{%
    \usepackage{parskip}
  }{% else
    \setlength{\parindent}{0pt}
    \setlength{\parskip}{6pt plus 2pt minus 1pt}}
}{% if KOMA class
  \KOMAoptions{parskip=half}}
\makeatother
\usepackage{xcolor}
\usepackage[heightrounded]{geometry}
\setlength{\emergencystretch}{3em} % prevent overfull lines
\setcounter{secnumdepth}{-\maxdimen} % remove section numbering
% Make \paragraph and \subparagraph free-standing
\ifx\paragraph\undefined\else
  \let\oldparagraph\paragraph
  \renewcommand{\paragraph}[1]{\oldparagraph{#1}\mbox{}}
\fi
\ifx\subparagraph\undefined\else
  \let\oldsubparagraph\subparagraph
  \renewcommand{\subparagraph}[1]{\oldsubparagraph{#1}\mbox{}}
\fi

\usepackage{color}
\usepackage{fancyvrb}
\newcommand{\VerbBar}{|}
\newcommand{\VERB}{\Verb[commandchars=\\\{\}]}
\DefineVerbatimEnvironment{Highlighting}{Verbatim}{commandchars=\\\{\}}
% Add ',fontsize=\small' for more characters per line
\newenvironment{Shaded}{}{}
\newcommand{\AlertTok}[1]{\textcolor[rgb]{1.00,0.33,0.33}{\textbf{#1}}}
\newcommand{\AnnotationTok}[1]{\textcolor[rgb]{0.42,0.45,0.49}{#1}}
\newcommand{\AttributeTok}[1]{\textcolor[rgb]{0.84,0.23,0.29}{#1}}
\newcommand{\BaseNTok}[1]{\textcolor[rgb]{0.00,0.36,0.77}{#1}}
\newcommand{\BuiltInTok}[1]{\textcolor[rgb]{0.84,0.23,0.29}{#1}}
\newcommand{\CharTok}[1]{\textcolor[rgb]{0.01,0.18,0.38}{#1}}
\newcommand{\CommentTok}[1]{\textcolor[rgb]{0.42,0.45,0.49}{#1}}
\newcommand{\CommentVarTok}[1]{\textcolor[rgb]{0.42,0.45,0.49}{#1}}
\newcommand{\ConstantTok}[1]{\textcolor[rgb]{0.00,0.36,0.77}{#1}}
\newcommand{\ControlFlowTok}[1]{\textcolor[rgb]{0.84,0.23,0.29}{#1}}
\newcommand{\DataTypeTok}[1]{\textcolor[rgb]{0.84,0.23,0.29}{#1}}
\newcommand{\DecValTok}[1]{\textcolor[rgb]{0.00,0.36,0.77}{#1}}
\newcommand{\DocumentationTok}[1]{\textcolor[rgb]{0.42,0.45,0.49}{#1}}
\newcommand{\ErrorTok}[1]{\textcolor[rgb]{1.00,0.33,0.33}{\underline{#1}}}
\newcommand{\ExtensionTok}[1]{\textcolor[rgb]{0.84,0.23,0.29}{\textbf{#1}}}
\newcommand{\FloatTok}[1]{\textcolor[rgb]{0.00,0.36,0.77}{#1}}
\newcommand{\FunctionTok}[1]{\textcolor[rgb]{0.44,0.26,0.76}{#1}}
\newcommand{\ImportTok}[1]{\textcolor[rgb]{0.01,0.18,0.38}{#1}}
\newcommand{\InformationTok}[1]{\textcolor[rgb]{0.42,0.45,0.49}{#1}}
\newcommand{\KeywordTok}[1]{\textcolor[rgb]{0.84,0.23,0.29}{#1}}
\newcommand{\NormalTok}[1]{\textcolor[rgb]{0.14,0.16,0.18}{#1}}
\newcommand{\OperatorTok}[1]{\textcolor[rgb]{0.14,0.16,0.18}{#1}}
\newcommand{\OtherTok}[1]{\textcolor[rgb]{0.44,0.26,0.76}{#1}}
\newcommand{\PreprocessorTok}[1]{\textcolor[rgb]{0.84,0.23,0.29}{#1}}
\newcommand{\RegionMarkerTok}[1]{\textcolor[rgb]{0.42,0.45,0.49}{#1}}
\newcommand{\SpecialCharTok}[1]{\textcolor[rgb]{0.00,0.36,0.77}{#1}}
\newcommand{\SpecialStringTok}[1]{\textcolor[rgb]{0.01,0.18,0.38}{#1}}
\newcommand{\StringTok}[1]{\textcolor[rgb]{0.01,0.18,0.38}{#1}}
\newcommand{\VariableTok}[1]{\textcolor[rgb]{0.89,0.38,0.04}{#1}}
\newcommand{\VerbatimStringTok}[1]{\textcolor[rgb]{0.01,0.18,0.38}{#1}}
\newcommand{\WarningTok}[1]{\textcolor[rgb]{1.00,0.33,0.33}{#1}}

\providecommand{\tightlist}{%
  \setlength{\itemsep}{0pt}\setlength{\parskip}{0pt}}\usepackage{longtable,booktabs,array}
\usepackage{calc} % for calculating minipage widths
% Correct order of tables after \paragraph or \subparagraph
\usepackage{etoolbox}
\makeatletter
\patchcmd\longtable{\par}{\if@noskipsec\mbox{}\fi\par}{}{}
\makeatother
% Allow footnotes in longtable head/foot
\IfFileExists{footnotehyper.sty}{\usepackage{footnotehyper}}{\usepackage{footnote}}
\makesavenoteenv{longtable}
\usepackage{graphicx}
\makeatletter
\def\maxwidth{\ifdim\Gin@nat@width>\linewidth\linewidth\else\Gin@nat@width\fi}
\def\maxheight{\ifdim\Gin@nat@height>\textheight\textheight\else\Gin@nat@height\fi}
\makeatother
% Scale images if necessary, so that they will not overflow the page
% margins by default, and it is still possible to overwrite the defaults
% using explicit options in \includegraphics[width, height, ...]{}
\setkeys{Gin}{width=\maxwidth,height=\maxheight,keepaspectratio}
% Set default figure placement to htbp
\makeatletter
\def\fps@figure{htbp}
\makeatother

\usepackage{fvextra}
\DefineVerbatimEnvironment{Highlighting}{Verbatim}{breaklines,commandchars=\\\{\}}
\KOMAoption{captions}{tableheading}
\makeatletter
\@ifpackageloaded{tcolorbox}{}{\usepackage[skins,breakable]{tcolorbox}}
\@ifpackageloaded{fontawesome5}{}{\usepackage{fontawesome5}}
\definecolor{quarto-callout-color}{HTML}{909090}
\definecolor{quarto-callout-note-color}{HTML}{0758E5}
\definecolor{quarto-callout-important-color}{HTML}{CC1914}
\definecolor{quarto-callout-warning-color}{HTML}{EB9113}
\definecolor{quarto-callout-tip-color}{HTML}{00A047}
\definecolor{quarto-callout-caution-color}{HTML}{FC5300}
\definecolor{quarto-callout-color-frame}{HTML}{acacac}
\definecolor{quarto-callout-note-color-frame}{HTML}{4582ec}
\definecolor{quarto-callout-important-color-frame}{HTML}{d9534f}
\definecolor{quarto-callout-warning-color-frame}{HTML}{f0ad4e}
\definecolor{quarto-callout-tip-color-frame}{HTML}{02b875}
\definecolor{quarto-callout-caution-color-frame}{HTML}{fd7e14}
\makeatother
\makeatletter
\@ifpackageloaded{caption}{}{\usepackage{caption}}
\AtBeginDocument{%
\ifdefined\contentsname
  \renewcommand*\contentsname{Table of contents}
\else
  \newcommand\contentsname{Table of contents}
\fi
\ifdefined\listfigurename
  \renewcommand*\listfigurename{List of Figures}
\else
  \newcommand\listfigurename{List of Figures}
\fi
\ifdefined\listtablename
  \renewcommand*\listtablename{List of Tables}
\else
  \newcommand\listtablename{List of Tables}
\fi
\ifdefined\figurename
  \renewcommand*\figurename{Figure}
\else
  \newcommand\figurename{Figure}
\fi
\ifdefined\tablename
  \renewcommand*\tablename{Table}
\else
  \newcommand\tablename{Table}
\fi
}
\@ifpackageloaded{float}{}{\usepackage{float}}
\floatstyle{ruled}
\@ifundefined{c@chapter}{\newfloat{codelisting}{h}{lop}}{\newfloat{codelisting}{h}{lop}[chapter]}
\floatname{codelisting}{Listing}
\newcommand*\listoflistings{\listof{codelisting}{List of Listings}}
\makeatother
\makeatletter
\makeatother
\makeatletter
\@ifpackageloaded{caption}{}{\usepackage{caption}}
\@ifpackageloaded{subcaption}{}{\usepackage{subcaption}}
\makeatother
\makeatletter
\@ifpackageloaded{tcolorbox}{}{\usepackage[skins,breakable]{tcolorbox}}
\makeatother
\makeatletter
\@ifundefined{shadecolor}{\definecolor{shadecolor}{rgb}{.97, .97, .97}}{}
\makeatother
\makeatletter
\@ifundefined{codebgcolor}{\definecolor{codebgcolor}{HTML}{D3D3D3}}{}
\makeatother
\makeatletter
\ifdefined\Shaded\renewenvironment{Shaded}{\begin{tcolorbox}[enhanced, colback={codebgcolor}, breakable, boxrule=0pt, frame hidden, sharp corners]}{\end{tcolorbox}}\fi
\makeatother
\ifLuaTeX
  \usepackage{selnolig}  % disable illegal ligatures
\fi
\usepackage{bookmark}

\IfFileExists{xurl.sty}{\usepackage{xurl}}{} % add URL line breaks if available
\urlstyle{same} % disable monospaced font for URLs
\hypersetup{
  colorlinks=true,
  linkcolor={blue},
  filecolor={Maroon},
  citecolor={Blue},
  urlcolor={Blue},
  pdfcreator={LaTeX via pandoc}}

\author{}
\date{}

\begin{document}

\RecustomVerbatimEnvironment{verbatim}{Verbatim}{
  showspaces = false,
  showtabs = false,
  breaksymbolleft={},
  breaklines
}

\renewcommand*\contentsname{Table of contents}
{
\hypersetup{linkcolor=}
\setcounter{tocdepth}{1}
\tableofcontents
}
\section{Documenting code}\label{documenting-code}

Documenting your code is crucial for both your future self and anyone
else who might want to work with your code. You want to document your
code with as much detail as you would fill out a lab book as your
documentation will help others (and your future self) understand the
purpose, functionality, and usage of your code.

\href{https://www.britishecologicalsociety.org/wp-content/uploads/2017/12/guide-to-reproducible-code.pdf}{A
Guide to Reproducible Code in Ecology and Evolution} gives detailed
information on how to organize project folders and how to write clear
and reproducible code. The examples are mainly based on R but are
general enough to apply to other computational languages (and scientific
disciplines).

\begin{tcolorbox}[enhanced jigsaw, rightrule=.15mm, breakable, opacitybacktitle=0.6, bottomrule=.15mm, colback=white, left=2mm, colframe=quarto-callout-note-color-frame, leftrule=.75mm, opacityback=0, title=\textcolor{quarto-callout-note-color}{\faInfo}\hspace{0.5em}{Note}, arc=.35mm, colbacktitle=quarto-callout-note-color!10!white, toprule=.15mm, titlerule=0mm, toptitle=1mm, coltitle=black, bottomtitle=1mm]

The information in this section is not part of the actual tutorial but
was added to give you a starting point for how to document your code.

If you follow the in-person tutorial you might want to start by
recording your notes using a plain text editor but feel free to explore
the more advanced options after the tutorial.

\end{tcolorbox}

\section{Choose Your Editor}\label{choose-your-editor}

\subsection{Plain text editor}\label{plain-text-editor}

When documenting code, its best to avoid visual editors, such as Word,
as these editors are not designed for writing code and easily destroy
the formatting by for example changing ` to ', which when writing code
is quite a big difference.

Instead you can use a plain text editor, such as TextEdit (Mac) or
Notepad (Windows). This is the easiest to get started but you will loose
some functionality, such as separating the code from comments, adding
headers or writing text in bold.

Alternatives, that offer more functionality, are for example RStudio or
VScode.

\subsection{Rmarkdown in RStudio}\label{rmarkdown-in-rstudio}

RMarkdown is an extension of Markdown (more on Markdown in a second)
that allows you to integrate R code directly into your documentation.

If you have not installed R and Rstudio, follow
\href{https://rstudio-education.github.io/hopr/starting.html}{these
instructions}.

In RStudio you can create an R Markdown File by:

\begin{itemize}
\tightlist
\item
  In RStudio, go to File -\textgreater{} New File -\textgreater{} R
  Markdown
\item
  Choose a title, author, and output format
\item
  After you are done writing your documentation you can knit the
  document into an HTML, PDF or word document:

  \begin{itemize}
  \tightlist
  \item
    Click the ``Knit'' button to render your R Markdown document into
    the chosen output format.
  \end{itemize}
\end{itemize}

For more information visit
\href{https://rmarkdown.rstudio.com/lesson-1.html}{the RMarkdown
tutorial}.

\subsection{Quarto in Rstudio}\label{quarto-in-rstudio}

Quarto is an alternative to RMarkdown for creating dynamic documents in
RStudio that can be read by other editors, such as VScode. Compared to
RMarkdown it provides enhanced features for document creation and
includes many more built in output formats (and many more options for
customizing each format).

It is installed by default on newer R installations. If you do not have
R and Rstudio installed, follow
\href{https://rstudio-education.github.io/hopr/starting.html}{these
instructions}.

\begin{itemize}
\tightlist
\item
  In RStudio, go to File -\textgreater{} New File -\textgreater{} Quarto
  document
\item
  Choose a title, author, and output format
\item
  Render the Document:

  \begin{itemize}
  \tightlist
  \item
    Click the ``Render'' button to render your R Markdown document into
    the chosen output format.
  \end{itemize}
\end{itemize}

For more information (and more functionality) visit
\href{https://quarto.org/docs/get-started/hello/rstudio.html}{the Quarto
website}.

\subsection{VSCode}\label{vscode}

Visual Studio Code (VSCode) is a versatile and user-friendly code
editor. It provides excellent support for various programming languages,
extensions, and a built-in terminal but might take a bit of work to
setup to work with different computational languages. VSCode might take
a bit longer to setup than RStudio but offers more flexibility due to
various extensions that users can install.

\begin{enumerate}
\def\labelenumi{\arabic{enumi}.}
\tightlist
\item
  Installation:

  \begin{itemize}
  \tightlist
  \item
    Download and install VSCode from
    \href{https://code.visualstudio.com/}{here}.
  \end{itemize}
\item
  Extensions:

  \begin{itemize}
  \tightlist
  \item
    Install extensions relevant to your programming language (e.g.,
    Python, R). These extensions enhance code highlighting and provide
    additional features.
  \end{itemize}
\end{enumerate}

\section{Markdown for Documentation}\label{markdown-for-documentation}

Markdown is a lightweight markup language that is easy to read and
write. It allows you to add formatting elements, such as headers, to
plain text documents.

\textbf{Headers:}

Use \texttt{\#} to add a header and separate different sections of your
documentation. The more \texttt{\#} symbols you use after each other,
the smaller the header will be. When writing a header make sure to
always put a space between the \texttt{\#} and the header name.

\begin{Shaded}
\begin{Highlighting}[]
\FunctionTok{\# Main Header}
\FunctionTok{\#\# Subheader}
\end{Highlighting}
\end{Shaded}

\textbf{Lists:}

Use \texttt{-} or \texttt{*} for unordered lists and numbers for ordered
lists.

Ordered lists are created by using numbers followed by periods. The
numbers do not have to be in numerical order, but the list should start
with the number one.

\begin{Shaded}
\begin{Highlighting}[]
\SpecialStringTok{1. }\NormalTok{First item}
\SpecialStringTok{2. }\NormalTok{Second item}
\SpecialStringTok{3. }\NormalTok{Third item}
\SpecialStringTok{4. }\NormalTok{Fourth item }
\end{Highlighting}
\end{Shaded}

\begin{Shaded}
\begin{Highlighting}[]
\SpecialStringTok{1. }\NormalTok{First item}
\SpecialStringTok{2. }\NormalTok{Second item}
\SpecialStringTok{3. }\NormalTok{Third item}
\SpecialStringTok{    1. }\NormalTok{Indented item}
\SpecialStringTok{    2. }\NormalTok{Indented item}
\SpecialStringTok{4. }\NormalTok{Fourth item }
\end{Highlighting}
\end{Shaded}

Unordered lists are created using dashes (\texttt{-}), asterisks
(\texttt{*}), or plus signs (\texttt{+}) in front of line items. Indent
one or more items to create a nested list.

\begin{Shaded}
\begin{Highlighting}[]
\SpecialStringTok{{-} }\NormalTok{First item}
\SpecialStringTok{{-} }\NormalTok{Second item}
\SpecialStringTok{{-} }\NormalTok{Third item}
\SpecialStringTok{{-} }\NormalTok{Fourth item }
\end{Highlighting}
\end{Shaded}

\begin{Shaded}
\begin{Highlighting}[]
\SpecialStringTok{ {-} }\NormalTok{First item}
\SpecialStringTok{{-} }\NormalTok{Second item}
\SpecialStringTok{{-} }\NormalTok{Third item}
\SpecialStringTok{    {-} }\NormalTok{Indented item}
\SpecialStringTok{    {-} }\NormalTok{Indented item}
\SpecialStringTok{{-} }\NormalTok{Fourth item }
\end{Highlighting}
\end{Shaded}

You can also combine ordered with unordered lists:

\begin{Shaded}
\begin{Highlighting}[]
\SpecialStringTok{1. }\NormalTok{First item}
\SpecialStringTok{2. }\NormalTok{Second item}
\SpecialStringTok{3. }\NormalTok{Third item}
\SpecialStringTok{    {-} }\NormalTok{Indented item}
\SpecialStringTok{    {-} }\NormalTok{Indented item}
\SpecialStringTok{4. }\NormalTok{Fourth item}
\end{Highlighting}
\end{Shaded}

\textbf{Code Blocks:}

Enclose code snippets in triple backticks followed by the computational
language, i.e.~bash or python, used.

\begin{Shaded}
\begin{Highlighting}[]
\InformationTok{\textasciigrave{}\textasciigrave{}\textasciigrave{}bash}
\FunctionTok{grep} \StringTok{"control"}\NormalTok{ downloads/Experiment1.txt}
\InformationTok{\textasciigrave{}\textasciigrave{}\textasciigrave{}}
\end{Highlighting}
\end{Shaded}

\textbf{Links:}

You can easily add links to external resources or within your
documentation as follows:

\begin{Shaded}
\begin{Highlighting}[]
\CommentTok{[}\OtherTok{Link Text}\CommentTok{](https://www.example.com)}
\end{Highlighting}
\end{Shaded}

\textbf{Emphasis:}

You can use \texttt{*} or \texttt{\_} to write italic and \texttt{**} or
\texttt{\_\_} for bold text.

\begin{Shaded}
\begin{Highlighting}[]
\NormalTok{*italic*}
\NormalTok{**bold**}
\end{Highlighting}
\end{Shaded}

\textbf{Pictures}

You can also add images to your documentation as follows:

\begin{Shaded}
\begin{Highlighting}[]
\AlertTok{![Alt Text](path/to/your/image.jpg)}
\end{Highlighting}
\end{Shaded}

Here, replace \texttt{Alt\ Text} with a descriptive alternative text for
your image, and \texttt{path/to/your/image.jpg} with the actual path or
URL of your image.

\textbf{Tables}

Tables can be useful for organizing information. Here's a simple table:

\begin{Shaded}
\begin{Highlighting}[]
\NormalTok{| Header 1 | Header 2 |}
\NormalTok{| {-}{-}{-}{-}{-}{-}{-}{-}{-}| {-}{-}{-}{-}{-}{-}{-}{-}{-}|}
\NormalTok{| Content 1| Content 2|}
\NormalTok{| Content 3| Content 4|}
\end{Highlighting}
\end{Shaded}




\end{document}
